Running the program 
===================

Inside FATSimulator folder there are the files: 
*Makefile*, *filesys.h *, * filesys.c* and *trace*\**.txt*.

You can compile and have the example run, by executing:
        $ make
        $ ./shell 
    

Now the FATsimulated files can be inspected, and the output of the shell
can be seen on the screen. The shell C file simulates a series of terminal 
commands the would be content of the disk are written to a series of files 
that can be inspected with a hex editor.

Running details 
=======================

When running *./shell* the user is informed about what functions/terminal 
commands will be demonstrated. While the results are the same as the ones that could be
expected from a real Unix File System some of the implementations differ
slightly. The biggest change beeing that *mychdir* changes a integer
that has the position of the current directory block instead of
modfiying a *direntry*\_*t* pointer. Each function that has been listed
as a requirement prints *function* start and stop so that the user
knows what operations have been done. Given that lots text is printed on
the screen some comments are also displayed in order to help the user
better understand what is goig to happen.



Details on the inner workings of particular functions. 
------------------------------------------------------

myfopen() 
---------

If the a file with the same name if it already exists it just opens it. 
If the directories specified in the path do not exist then
they are created and then the file is made.

mychdir() 
---------

It has a currentDirectory integer that represents the block number of
current directory. When called it changes the value of variable pointing
to the current directory to point to the new current directory.

mymkdir() 
---------
Some parts of the memory blocks ocupied by directories are not 
initialized and some form of clutter might appear. However it is 
tied to the unused space of a directory and the directory structure 
in memory can always be seen at the begginig of a new blocks. After the 
deletion of a directory or a entry. Basically, it is not a bug but a feature.
